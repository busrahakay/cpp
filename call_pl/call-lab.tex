\documentclass{article}
\usepackage[margin=2cm]{geometry}
\usepackage[T1]{fontenc}
\usepackage[utf8]{inputenc}
\usepackage{color}
\usepackage{listings}
\usepackage{csquotes}

\input{listings.tex}

\title{Wyrażenia wywołania}

\author{dr inż.~Ireneusz Szcześniak}

\date{jesień 2018}

\begin{document}

\maketitle

\section{Zadanie}

Stworzyć kolejkę priorytetową liczb całkowitych.  Włożyć liczby 2, 3,
1, a następnie wyciągnąć te liczby i wypisać.

\subsection{Rozwiązanie}

\lstinputlisting{pq.cc}

\section{Wywołanie funkcji przez wskaźnik}

Wyrażenie, które jest tylko nazwą funkcji oznacza pobranie adresu
funkcji.  Używając tego adresu możemy wywołać funkcję.  Jedyne
operacje możliwe na wskaźniku do funkcji to: pobranie adresu funkcji i
wywołanie funkcji.

\lstinputlisting{function.cc}

\section{Zadanie}

Domyślnie kolejka priorytetowa zwraca największy element.  Przerobić
rozwiązanie wyżej z użyciem własnej funkcji porównującej elementy,
żeby kolejka zwracała najmniejszy element.

\subsection{Rozwiązanie}

\lstinputlisting{pq_foo.cc}

\section{Zadanie}

Domyślnie kolejka priorytetowa zwraca największy element, bo do
porównania używa klasy funktora \code{std::less<T>}.  Użyć klasy
\code{std::greater<T>}, żeby kolejka zwracała najmniejszy element.

\subsection{Rozwiązanie}

\lstinputlisting{pq_ro.cc}

\section{Zadanie}

Do sortowania kolejki użyć funktora.  Kolejka priorytetowa powinna
zwracać najmniejszy element.

\subsection{Rozwiązanie}

\lstinputlisting{pq_fo.cc}

\section{Zadanie}

Do sortowania kolejki użyć funktora, ale tym razem umożliwić w czasie
uruchomienia wybór kierunku sortowania kolejki: malejący albo rosnący
w zależności od argumentu wywołania konstruktora funktora.

\subsection{Rozwiązanie}

\lstinputlisting{pq_fo2.cc}

\section{Zadanie}

Do sortowania kolejki użyć wyrażenia lambda.  Kolejka priorytetowa
powinna zwracać najmniejszy element.

\subsection{Rozwiązanie}

\lstinputlisting{pq_lambda.cc}

\section{Zadanie}

Do sortowania kolejki użyć wyrażenia lambda, ale tym razem umożliwić w
czasie uruchomienia wybór kierunku sortowania kolejki: malejący albo
rosnący w zależności od argumentu wyrażenia lambda.

\subsection{Rozwiązanie}

\lstinputlisting{pq_lambda2.cc}

\end{document}
