\documentclass{article}
\usepackage[margin=2cm]{geometry}
\usepackage[T1]{fontenc}
\usepackage[utf8]{inputenc}
\usepackage{color}
\usepackage{listings}
\usepackage{csquotes}

\input{listings.tex}

\title{Kontenery}

\author{dr inż.~Ireneusz Szcześniak}

\date{jesień 2017}

\begin{document}

\maketitle

\section{Zadanie 1}

Napisać szablonowy kontener \code{range}, który będzie generował
żądany ciąg liczb całkowitych przy użyciu pętli \code{for}.  To jest
przykładowy plik \code{range.cc}:

\lstinputlisting{range.cc}

\section{Rozwiązanie zadania}

Zawartość pliku \code{range.hpp}:

\lstinputlisting{range.hpp}

\section{Zadanie 2}

Napisać klasę \code{A} tylko do przenoszenia.  Stworzyć dwa zbiory
obiektów tej klasy.  Dodać jeden element do pierwszego zbioru, a potem
ten element bez kopiowania do drugiego zbioru.

\section{Rozwiązanie zadania}

\lstinputlisting{extract.cc}

\end{document}
